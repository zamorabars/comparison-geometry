\chapter{Curvature of curves}

\section{Total curvature}

Here we introduce the so called \emph{total curvature of curve}.
In general term \emph{curvature} is used for something that measures how 
a geometric object deviates from being a straight;
total curvature is not an exception --- as you will see if the total curvature of a curve is vanishing then the curve runs along a straight line.


Let $\alpha\:[a,b]\to\RR^3$ be a \emph{smooth} \emph{regular} curve --- smooth means that
the velocity vector $\alpha'(t)$ is defined and continuous with respect to $t$ and regular means that $\alpha'(t)\ne 0$ for any $t$.
If the curve $\alpha$ is closed then we assume in addition that $\alpha'(a)=\alpha'(b)$.

Denote by $\tau(t)$ the unit vector in the direction of $\alpha'(t)$;
that is, $\tau(t)=\tfrac{\alpha'(t)}{|\alpha'(t)|}$.
The $\tau\:[a,b]\to\mathbb{S}^2$ is an other curve which is called \emph{tangent indicatrix} of $\alpha$.
The length of $\tau$ is called \emph{total curvature of}~$\alpha$.

\begin{thm}{Exercise}
Show that the total curvature of any smooth closed regular curve $\alpha$ is at least $2\cdot\pi$.
Moreover, the equality holds if and only if $\alpha$ is a colased and conex curve that lies in a plane.
\end{thm}
